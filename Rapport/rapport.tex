\documentclass[a4paper,10pt, english]{article}
\usepackage[utf8]{inputenc}
\usepackage[T1]{fontenc}
\usepackage{graphicx}
\newcommand{\code}[1]{\texttt{#1}}
\newcommand{\chan}{\texttt{channel} }

% Title Page
\title{Autoencoder - Learning to represent signals spike by spike}
\author{Yu-Guan Hsieh, Martin Ruffel}


\begin{document}
\maketitle

\section*{Recap of our work}

We worked on the spiking autoencoder. We started from a standard spiking network trying to represent an input signal.
Then, we added recurrent weights to simulate the substraction of the estimate, and to force neurons' potentials
to encode the signal reconstruction error. 


We rederived the main equations and learning rules of the network, starting from the potential equations, 
and the loss function $L = ||x - \hat{x}||^2 + \nu \sum{r_{i}} + \mu \sum{r_{i}^2}$ which penalizes high 
firing rates. We followed the same path as in the article's appendix.

\begin{itemize}
 \item To simulate the reconstruction error, we want to have $\Omega = FD$
 \item Then, the potentials $V$ represent approximately the projection $F(x - \hat{x})$ of the error.
 \item We want neuron $i$ to spike if and only if its spike reduces the loss function $L$. Putting this 
 in equations, and identifying terms as membrane threshold and potential allows us to obtain the optimal 
 decoder $D = F^\top$.
 \item Since $V$ is proportionnal to reconstruction error, we want to minimize $V$. Therefore, we derive 
 a simple learning rule for $\Omega$ :
 \begin{center}
  {$\Delta\Omega_{i,j} = -V_{i} -\Omega_{i,j}$} when neuron j spikes
 \end{center}
Indeed, this corresponds to reset potential of all the neurons at each spike.
 \item Adding the L2 penalization term changes this learning rule because it adds term $-\mu r$ in potential $V$:
 \begin{center}
   \centering{$\Delta\Omega_{i,j} = -V_{i} - \mu r_i - \Omega_{i,j}  -\mu\delta_{i, j}$}
 \end{center}
 \item If we suppose that $F$ is updated a lot slower than $\Omega$, then some equations give us a similar 
 learning rule for $F$ :
 \begin{center}
  $\Delta F_{i,j} = (x - \hat{x}) - F$, what is approximated by $\Delta F_{i,j} = \alpha x - F$ when neuron i spikes
 \end{center}
\end{itemize}

With these learning rules, we have implemented in Python the spiking autoencoder, using the article's Appendix pseudo-code.
But when we launched some tests to see if these equations work, we obtained a lot of problems.
\begin{itemize}
 \item There were some little errors in pseudo-code, or unexplained modifications.
 \item The obtained network was very unstable with the given parameters. Indeed, after some training steps,
 only a few neurons were spiking almost at each time step, and the others did nothing. We spent a lot of 
 time trying to solve this, and modifying constants like $\lambda$, $\mu$, definition of $c$ to obtain coherent 
 results. After two weeks, we finally found some nice parameters.
 \item With these new parameters, the weights $F$ don't change a lot, and therefore tuning curves don't move through
 training to cover the whole input space.
 \item Reconstruction error doesn't seem to decrease through training.
 \item It is very hard to adjust parameters of input and network to obtain such nice reconstructions as in 
 the main paper.
 
\end{itemize}


The only satisfying results we obtained were that the firing rates decrease during the training : there are divided
by at least 4.

Since we didn't get nice results, we couldn't compare the outputs of the network to independent Poisson processes, and we didn't
try to feed the network with correlated inputs.


\end{document}          
